\newglossaryentry{design}{
name={Design Thinking},
description={Metodología de diseño centrada en las necesidades del usuario}
}

\newglossaryentry{API}{
name={API},
description={Application Programming Interface. Conjunto de funciones y datos que permiten a los desarrolladores interactuar con un sistema}
}

\newglossaryentry{mockup}{
name={mockup},
description={Prototipo de un diseño que se utiliza para probar la funcionalidad y el aspecto de un producto o servicio}
}

\newglossaryentry{milestone}{
name={milestone},
description={Objetivo o hito importante en el desarrollo de un proyecto}
}

\newglossaryentry{issue}{
name={issue},
description={Problema o error que se encuentra en un software}
}

\newglossaryentry{PR}{
name={Pull Request},
description={Solicitud para fusionar cambios de un repositorio de código a otro}
}

\newglossaryentry{cherry-picking}{
name={cherry-picking},
description={Práctica de seleccionar cambios específicos de una rama de código para fusionarlos con otra}
}

\newglossaryentry{commit}{
name={commit},
description={Cambio de código que se realiza en un repositorio}
}

\newglossaryentry{git}{
name={Git},
description={Sistema de control de versiones distribuido}
}

\newglossaryentry{github}{
name={GitHub},
description={Plataforma de alojamiento de repositorios de código}
}

\newglossaryentry{jira}{
name={Jira},
description={Software de gestión de proyectos y tareas}
}

\newglossaryentry{gitlab}{
name={GitLab},
description={Plataforma de alojamiento de repositorios de código con funciones de gestión de proyectos}
}

\newglossaryentry{framework}{
name={framework},
description={Biblioteca o conjunto de herramientas que proporciona una estructura para el desarrollo de software}
}

\newglossaryentry{python}{
name={python},
description={Lenguaje de programación interpretado de alto nivel}
}

\newglossaryentry{crawler}{
name={crawler},
description={Programa que navega por la web para recopilar información}
}

\newglossaryentry{HTML}{
name={HTML},
description={HyperText Markup Language. Lenguaje de marcado para la creación de páginas web}
}

\newglossaryentry{django}{
name={django},
description={Framework web para Python}
}

\newglossaryentry{flask}{
name={flask},
description={Microframework web para Python}
}

\newglossaryentry{WSGI}{
name={WSGI},
description={Web Server Gateway Interface. Interfaz que permite a los servidores web interactuar con frameworks web}
}

\newglossaryentry{SEO}{
name={SEO},
description={Search Engine Optimization. Conjunto de técnicas para mejorar el posicionamiento de un sitio web en los resultados de búsqueda}
}

\newglossaryentry{CSRF}{
name={CSRF},
description={Cross-Site Request Forgery. Ataque que permite a un atacante realizar acciones en nombre de un usuario sin su conocimiento}
}

\newglossaryentry{SQL}{
name={Structured Query Language},
description={Structured Query Language. Lenguaje de consulta estructurado para bases de datos relacionales}
}

\newglossaryentry{XSS}{
name={XSS},
description={Cross-Site Scripting. Inyección de código malicioso en una página web}
}

\newglossaryentry{frontend}{
name={frontend},
description={Parte de una aplicación web que se ejecuta en el navegador del usuario}
}

\newglossaryentry{base}{
name={Base de datos},
description={Repositorio de datos estructurados que se utilizan para almacenar información}
}

\newglossaryentry{http}{
name={HTTP},
description={HyperText Transfer Protocol. Protocolo de comunicación que se utiliza para transferir datos entre un servidor web y un cliente}
}

\newglossaryentry{JSON}{
name={JSON},
description={JavaScript Object Notation. Formato de datos ligero basado en JavaScript}
}

\newglossaryentry{CSS}{
name={CSS},
description={Cascading Style Sheets. Lenguaje de diseño para páginas web}
}

\newglossaryentry{script}{
name={script},
description={Código que se ejecuta en el navegador del usuario}
}

\newglossaryentry{dataset}{
name={dataset},
description={Conjunto de datos estructurados}
}

\newglossaryentry{ORM}{
name={ORM},
description={Object Relational Mapping. Arquitectura que permite a los desarrolladores interactuar con bases de datos relacionales utilizando objetos}
}

\newglossaryentry{singleton}{
name={singleton},
description={Diseño de patrón que garantiza que solo exista una instancia de una clase}
}

\newglossaryentry{tupla}{
name={tupla},
description={Colección de datos inmutable}
}

\newglossaryentry{test}{
name={test},
description={Proceso de verificar el funcionamiento correcto de un software}
}

\newglossaryentry{interfaz}{
name={interfaz},
description={Espacio de comunicación entre dos sistemas o usuario-máquina}
}

\newglossaryentry{CDN}{
name={CDN},
description={Content Delivery Network. Red de distribución de contenido que almacena y sirve contenido web desde servidores distribuidos}
}

\newglossaryentry{tag html}{
name={tag},
description={Etiqueta que define el contenido y el comportamiento de un elemento en una página web}
}

\newglossaryentry{AJAX}{
name={AJAX},
description={Asynchronous JavaScript and XML. Técnica que permite a las páginas web actualizar su contenido de forma asíncrona}
}

\newglossaryentry{JQuery}{
name={JQuery},
description={Librería JavaScript para facilitar el desarrollo de AJAX}
}

\newglossaryentry{endpoint}{
name={endpoint},
description={Punto final de una API}
}

\newglossaryentry{cookie}{
name={cookie},
description={Pequeña pieza de información que un servidor web almacena en el navegador del usuario}
}

\newglossaryentry{SVG}{
name={SVG},
description={Scalable Vector Graphics. Formato de imagen vectorial escalable}
}

\newglossaryentry{backend}{
name={backend},
description={Parte de una aplicación web que se ejecuta en el servidor}
}

\newglossaryentry{MVC}{
name={Model-View-Controller},
description={Model-View-Controller. Patrón de diseño que divide una aplicación web en tres componentes: modelo, vista y controlador}
}

\newglossaryentry{poetry}{
name={Poetry},
description={Gestor de paquetes para Python}
}

\newglossaryentry{event}{
name={event listener},
description={Función que se ejecuta cuando ocurre un evento}
}

\newglossaryentry{render}{
name={render},
description={Método que se utiliza para generar la respuesta HTML de una vista Django}
}

\newglossaryentry{middleware}{
name={middleware},
description={Software que se ejecuta entre el servidor web y la aplicación web}
}

\newglossaryentry{EC2}{
name={E2},
description={Elastic Compute Cloud. Servicio de Amazon Web Services que proporciona instancias de servidores virtuales}
}

\newglossaryentry{AWS}{
name={AWS},
description={Amazon Web Services. Plataforma de computación en la nube de Amazon}
}

\newglossaryentry{RDS}{
name={RDS},
description={Relational Database Service. Servicio de Amazon Web Services que proporciona bases de datos relacionales}
}

\newglossaryentry{CIDR}{
name={CIDR},
description={Classless Inter-Domain Routing. Método de direccionamiento IP que permite asignar bloques de direcciones IP más grandes}
}

\newglossaryentry{VPC}{
name={VPC},
description={Virtual Private Cloud. Red virtual aislada que se ejecuta en la nube}
}

\newglossaryentry{ACL}{
name={ACL},
description={Access Control List. Lista de control de acceso que permite definir permisos de acceso a recursos}
}

\newglossaryentry{VPN}{
name={VPN},
description={Virtual Private Network. Red privada virtual que permite conectar dispositivos de forma segura a través de Internet}
}

\newglossaryentry{SSH}{
name={Secure Shell},
description={Secure Shell .Protocolo de red seguro que permite conectarse a un servidor de forma remota}
}

\newglossaryentry{ELB}{
name={ELB},
description={Elastic Load Balancing. Servicio de Amazon Web Services que distribuye el tráfico entre varias instancias de EC2}
}

\newglossaryentry{DNS}{
name={DNS},
description={Domain Name System. Sistema de nombres de dominio que traduce los nombres de dominio en direcciones IP}
}

\newglossaryentry{IaaS}{
name={IaaS},
description={Infrastructure as a Service .Modelo de computación en la nube que permite a las empresas alquilar recursos informáticos, como servidores, almacenamiento y redes, a un proveedor de servicios en la nube}
}

\newglossaryentry{on-premises}{
name={on-premises},
description={Se refiere a los recursos informáticos que se encuentran en las instalaciones de una empresa}
}