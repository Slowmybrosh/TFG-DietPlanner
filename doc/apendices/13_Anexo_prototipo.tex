\chapter{Plataforma Web}
\section{Interfaz}
\begin{figure}[H]
    \begin{lstlisting}[style=consola]
    <head>
        <title>RecetApp</title>

        <!--Custom style css for more spaces-->
        <link rel="stylesheet" href="">

        <!-- Latest compiled and minified CSS -->
        <link rel="stylesheet" href="https://cdn.jsdelivr.net/npm/bootstrap@5.2.3/dist/css/bootstrap.min.css" integrity="sha384-rbsA2VBKQhggwzxH7pPCaAqO46MgnOM80zW1RWuH61DGLwZJEdK2Kadq2F9CUG65" crossorigin="anonymous">

        <!-- Latest compiled and minified JavaScript -->
        <script src="https://cdn.jsdelivr.net/npm/bootstrap@5.2.3/dist/js/bootstrap.min.js" integrity="sha384-cuYeSxntonz0PPNlHhBs68uyIAVpIIOZZ5JqeqvYYIcEL727kskC66kF92t6Xl2V" crossorigin="anonymous"></script>

        <!-- Latest compiled JQuery-->
        <script src="https://ajax.googleapis.com/ajax/libs/jquery/3.7.1/jquery.min.js"></script>
    </head>
    \end{lstlisting}
    \caption{Template de los ficheros fuente de la \gls{interfaz}}
    \label{sni:docker}
\end{figure}

\begin{figure}[H]
\begin{lstlisting}[style=consola]
    <body class="">
        <nav class="navbar navbar-expand-lg navbar-light background-nav border-bottom border-dark">
            <div class="container-fluid">
              <a class="navbar-brand" href="/">RecetApp</a>
              <button class="navbar-toggler" type="button" data-bs-toggle="collapse" data-bs-target="#navbarSupportedContent" aria-controls="navbarSupportedContent" aria-expanded="false" aria-label="Toggle navigation">
                <span class="navbar-toggler-icon"></span>
              </button>
              <div class="collapse navbar-collapse in" id="navbarSupportedContent">
                <ul class="navbar-nav me-auto mb-2 mb-lg-0">
                  <li class="nav-item">
                    <a class="nav-link active" aria-current="page" href="#">Home</a>
                  </li>
                  <li class="nav-item">
                    <a class="nav-link " aria-current="page" href="/guardadas/">Recetas guardadas</a>
                  </li>
                  <li class="nav-item">
                    <a class="nav-link" href="#">About</a>
                  </li>
                </ul>
                <form class="d-flex" action="/resultados/" method="POST">
                  
                  <input class="form-control me-2 border border-dark" type="search" placeholder="Nombre de la receta" aria-label="Search" name="search-recetas">
                  <button class="btn btn-success" type="submit">Buscar</button>
                  <input type="hidden" name="metodo" value="recetas">
                </form>
              </div>
            </div>
          </nav>
\end{lstlisting}
\caption{Template del navegador de la \gls{interfaz}}
\label{sni:interfaz}
\end{figure}

\begin{figure}[H]
\begin{lstlisting}[style=consola]
        <main class="background">
            <div class="container-fluid pt-3">
                <div class="row justify-content-center">
                <div class="col-lg-9 col-md-9 container-fluid">
                    
                    
                </div>
                <div class="col-lg-3 col-md-9 mt-5 pe-lg-3 d-flex flex-column overflow-auto">
                    
                    <h3 class="w-100 text-center">Algunas recetas</h3>
                    
                        <div class="card border rounded w-100" style="width: 18rem;">
                        <a class="link-unstyled" href="#">
                            <div class="card-body border border-dark background-card">
                            <h5 class="card-title">{{item.name}}</h5>
                            <a href="/receta/{{item.id}}" class="btn button-bg border border-dark">Ver receta</a>
                            </div>
                        </a>
                        </div>
                    
                    
                </div>
                </div>
            </div>
        </main>
    </body>
    <script src=""></script>
</html>
\end{lstlisting}
\caption{Template del contenedor principal la base de la \gls{interfaz}}
\label{sni:interfaz}
\end{figure}

\begin{figure}[H]
\begin{lstlisting}[style=consola]


    <div class="row mt-5">
        <div class="col-lg-2 col-none"></div>
            <div class="col-lg-8 col-12 justify-content-center">
                <h1 class="text-center">RecetApp</h1>
                <form class="form-control border border-dark" action="/resultados/" method="POST">
                    
                    <div class="d-inline-flex w-100">
                        <input id="buscador-ingredientes" class="form-control me-2" type="search" placeholder="Buscar Ingredientes" aria-label="Search">
                        <button id="enviarIngredientes" class="btn btn-success" type="submit">Buscar Recetas</button>
                        <div id="eliminarSeleccion" class="btn btn-danger ms-3">Borrar Seleccion</div>
                    </div>
                        <div class="container">
                            <table id="tabla" hidden class="table mt-5 container ingredientes">
                                <thead class="container">
                                    <tr class="justify-content">
                                        <th>Nombre</th>
                                        <th class="text-center">Seleccionado</th>
                                    </tr>
                                </thead>
                                <tbody id="tabla-ingredientes" class="">
                                </tbody>
                            </table>
                        </div> 
                    <input type="hidden" name="metodo" value="ingredientes">
                </form>
            <div class="col-lg-2 col-none"></div>
        </div>
    </div>

\end{lstlisting}
\caption{Bloque HTML del buscador}
\label{sni:buscador}
\end{figure}

\begin{figure}[H]
\begin{lstlisting}[style=consola]



<div class="container-fluid h-100 background">
    <div class="row d-flex-inline bg-light rounded rounded-sm background">
        <div class="col-lg-8 background">
            <div class="card mt-5 w-100">
                <img class="card-img border-bottom border-dark" src="" width="100%">
                <h1 class="card-title">{{receta_nombre}}</h1>
                <div class="ps-3 mt-5 card-body">
                    <h4>Ingredientes</h4>
                    <ul>
                    
                        <li>{{item}}</li>
                    
                    </ul>
                    <h4 class="mt-5">Pasos</h4>
                    <ol>
                        
                        <li><p>{{item}}</p></li>
                        
                    </ol>
                </div>
            </div>
        </div>
        <div class="col-lg background">
        <div class="container-fluid">
            <div class="row rounded-sm d-flex flex-column justify-content-center-100">
            <button id="guardarReceta" data-id={{receta_id}} class="btn save-btn flex-grow button-bg btn-outline-success btn-lg mt-5 mb-3 border border-dark" saved={{saved}} origin="receta">Guardar receta</button>
            <div id="compartirReceta" data-id={{receta_id}} class="btn share-btn flex-grow btn-lg btn-light border border-dark">Compartir</div>
            </div>
        </div>
        </div>
    </div>
</div>
</div>

\end{lstlisting}
\caption{Bloque HTML de la receta}
\label{sni:receta}
\end{figure}

\begin{figure}[H]
\begin{lstlisting}[style=consola]
    
    
        
            <h2><u>Recetas encontradas</u></h2>
            
                <div id="resultados" class="mt-1">
                    <div class="card mb-1 d-flex-inline flex-row row">
                        <div class="col-lg-8">
                            <a class="link-unstyled w-100" href="/receta/{{resultado.0}}">
                                <div class="card-img"></div>
                                <div class="card-body">
                                    <div class="card-title"><h4>{{ resultado.1 }}</h4></div>
                                    <div class="card-text">
                                        <div class="container d-flex-inline">
                                            <p class=""><u>Ingredientes utilizados</u></p>
                                            <ul>
                                                
                                                    <li>{{item}}</li>
                                                
                                            </ul>
                                        </div>
                                    </div>
                                </div>
                            </a>
                        </div>
                        <div class="col-lg container">
                            <div class="row rounded-sm d-flex flex-column justify-content-center-100 pe-3">
                              <button data-id={{resultado.0}} class="save-btn btn flex-grow button-bg btn-outline-success btn-lg mt-5 mb-3 border border-dark" saved={{resultado.3}} origin="resultado">Guardar receta</button>
                              <button id="compartirReceta" data-id={{resultado.0}} class="btn share-btn flex-grow btn-lg btn-light border border-dark">Compartir</button>
                            </div>
                          </div>
                    </div>
                </div>
            
        
            <h2>No se han encontrado recetas</h2>
        
    
\end{lstlisting}
\caption{Bloque HTML de la lista de resultados}
\label{sni:resultados}
\end{figure}

\begin{figure}[H]
\begin{lstlisting}[style=consola]
$(document).ready(function() {
    $('#buscador-ingredientes').on('input', function() {
        var query = $(this).val();
        $.ajax({
            url: '/buscar/',
            data: {'query': query},
            dataType: 'json',
            success: function(response) {
                BuscarIngrediente(response)
            }
        });
    });
});
\end{lstlisting}
\caption{Evento del buscador}
\label{sni:evnt-buscador}
\end{figure}

\begin{figure}[H]
\begin{lstlisting}[style=consola]
function BuscarIngrediente(response) {
    if(ObtenerCookie("IDs") == null){
        var json = {
            seleccionados: []
        };
        crearCookie("IDs",JSON.stringify(json),1);
    }
    $('#tabla').removeAttr('hidden');
    var tabla = $('#tabla-ingredientes');
    if(response == null){
        tabla.empty();
    } else{
        var resultados = response.resultados;
        tabla.empty();
        for (var i = 0; i < resultados.length; i++) {
            var fila = $('<tr data-id="'+resultados[i].id+'" onclick="SeleccionarIngrediente(this)">');
            fila.append($('<td class="">').text(resultados[i].nombre));
            var td = $('<td class="text-center">');
            var svg = '<svg';
            if(isSeleccionado(resultados[i].id) == false){
                svg += " hidden"
            }
            svg += <tag del SVG>
            td.append(svg);
            fila.append(td);
            tabla.append(fila);
        }
    }
}
\end{lstlisting}
\caption{Función para buscar dinámicamente los ingredientes}
\label{sni:buscadorIngredientes}
\end{figure}

\begin{figure}[H]
\begin{lstlisting}[style=consola]
function SeleccionarIngrediente(row){
    var svg = row.querySelector('svg');
    if(svg.hasAttribute('hidden')){
        svg.removeAttribute('hidden');
        var json = ObtenerCookie("IDs");
        if(json != null){
            var lista = JSON.parse(json);
            lista.seleccionados.push(row.getAttribute('data-id'));
            crearCookie("IDs",JSON.stringify(lista))
        }
    }
    else{
        svg.setAttribute('hidden','true');
        var json = ObtenerCookie("IDs");
        if(json != null){
            var lista = JSON.parse(json);
            lista = eliminarID(row.getAttribute('data-id'),lista);
            crearCookie("IDs",JSON.stringify(lista))
        }
    }
}
\end{lstlisting}
\caption{Función para seleccionar el ingrediente de la lista}
\label{sni:seleccion}
\end{figure}

\begin{figure}[H]
\begin{lstlisting}[style=consola]
function ObtenerCookie(cookieName) {
    var cookieValue = null;
    var cookieString = document.cookie;
    var cookies = cookieString.split(';');

    for (var i = 0; i < cookies.length; i++) {
        var cookie = cookies[i].trim();

        if (cookie.startsWith(cookieName + '=')) {
        cookieValue = cookie.substring(cookieName.length + 1);
        break;
        }
}

return cookieValue;
}
\end{lstlisting}
\caption{Función para obtener la cookie}
\label{sni:cookie-Obtener}
\end{figure}

\begin{figure}[H]
\begin{lstlisting}[style=consola]
function crearCookie(nombre, valor, diasExpiracion) {
    var fechaExpiracion = new Date();
    fechaExpiracion.setDate(fechaExpiracion.getDate() + diasExpiracion);
    var cookieString = nombre + '=' + valor + '; expires=' + fechaExpiracion.toUTCString() + '; path=/';
    document.cookie = cookieString;
}
\end{lstlisting}
\caption{Función para crear la cookie}
\label{sni:cookie-Crear}
\end{figure}

\begin{figure}[H]
\begin{lstlisting}[style=consola]
function eliminarID(id, lista){
    var indice = lista.seleccionados.indexOf(String(id));
    if (indice !== -1) {
        lista.seleccionados.splice(indice, 1);
    }
    return lista
}
\end{lstlisting}
\caption{Función para eliminar un ingrediente de seleccionados}
\label{sni:seleccionados-eliminar}
\end{figure}

\begin{figure}[H]
\begin{lstlisting}[style=consola]
    function eliminarID(id, lista){
    var indice = lista.seleccionados.indexOf(String(id));
    if (indice !== -1) {
        lista.seleccionados.splice(indice, 1);
    }
    return lista
}
\end{lstlisting}
\caption{Función para eliminar un ingrediente de seleccionados}
\label{sni:seleccionados-eliminar}
\end{figure}

\section{Framework}

\begin{figure}[H]
\begin{lstlisting}[style=consola]
class Ingredients(models.Model):
    id = models.IntegerField(primary_key=True)
    name = models.CharField(max_length=255)

    class Meta:
        db_table = 'ingredient'

class Recipes(models.Model):
    id = models.AutoField(primary_key=True)
    name = models.CharField(max_length=255)
    steps = models.TextField()
    number_of_ingredients = models.IntegerField(null=True)

    class Meta:
        db_table = 'recipes'

class RecipeIngredient(models.Model):
    recipe_id = models.ForeignKey(Recipes, on_delete=models.CASCADE)
    ingredient_id = models.ForeignKey(Ingredients, on_delete=models.CASCADE)

    class Meta:
        db_table = 'recipeingredient'
        constraints = [
            models.UniqueConstraint(fields=['recipe_id', 'ingredient_id'], name='unique_recipe_ingredient'),
        ]

\end{lstlisting}
\caption{Fichero con los modelos de \gls{Django}}
\label{sni:django-modelos}
\end{figure}

\begin{figure}[H]
\begin{lstlisting}[style=consola]
from django.urls import path
from . import views

urlpatterns = [
    path("",views.index, name="index"),
    path("buscar/", views.buscar, name="buscar_resultados"),
    path("resultados/", views.resultados, name="resultados"), #Endpoint para buscar recetas
    path("receta/<int:id>", views.detalle_receta, name="detalle_receta"), #Endpoint para buscar detalles de una receta
    path("guardadas/", views.guardadas, name="guardadas")
]
\end{lstlisting}
\caption{Fichero con los \gls{endpoint} de \gls{Django}}
\label{sni:django-urls}
\end{figure}

\begin{figure}[H]
\begin{lstlisting}[style=consola]
def index(request):
    """
    Endpoint para el home
    """
    context = {
        "top": buscador.getRandom(5)
    }
    return render(request, 'buscador.html', context)
\end{lstlisting}
\caption{Punto de acceso a la página principal}
\label{sni:django-index}
\end{figure}

\begin{figure}[H]
    \begin{lstlisting}[style=consola]
    def buscar(request):
        """
        Endpoint para buscar los ingredientes en base a una query
        """
        resultados_lista = []
        if(request.method == 'GET'):
            query = request.GET.get('query','')
            resultados = buscador.buscarIngredienteNombre(query)
            for resultado in resultados:
                resultado_dict = {
                    'id': resultado[0],
                    'nombre': resultado[1]
                }
                resultados_lista.append(resultado_dict)
        return JsonResponse({'resultados': resultados_lista})
    \end{lstlisting}
    \caption{}
    \label{sni:django-buscar}
\end{figure}

\begin{figure}[H]
    \begin{lstlisting}[style=consola]
        def resultados(request):
            """
            Endpoint para buscar recetas en base a los ingredientes seleccionados
            """
            if request.method == 'POST':
                metodo = request.POST.get("metodo")
                if request.COOKIES.get("recetas"):
                    guardadas_json = json.loads(request.COOKIES.get("recetas"))
                    guardadas  = [eval(i) for i in guardadas_json]
                else:
                    guardadas = []
                if metodo == "ingredientes":
                    if request.COOKIES.get("IDs"):
                        seleccionados_json = json.loads(request.COOKIES.get("IDs"))
                        if seleccionados_json:
                            IDs = list(map(int, seleccionados_json["seleccionados"]))
                            lista_recetas = buscador.buscarRecetasIngredientes(IDs)
                            recetas = []
                            for receta in lista_recetas:
                                recetas.append((receta[0],receta[1],receta[2],receta[0] in guardadas))
                            context = {
                                "top": buscador.getRandom(5),
                                "resultados": recetas
                            }
                        else:
                            context = {
                                "top": buscador.getRandom(5)
                            }
                    else:
                        return redirect("index")
                if metodo == "recetas":
                    query = request.POST.get('search-recetas')
                    lista_recetas = buscador.buscarRecetasNombre(query)
                    recetas = []
                    for receta in lista_recetas:
                        recetas.append((receta[0],receta[1],receta[2],receta[0] in guardadas))
                    context = {
                        "top": buscador.getRandom(5),
                        "resultados": recetas
                    }
                return render(request,'resultados.html',context)
            else:
                return redirect("index")
    \end{lstlisting}
    \caption{\gls{endpoint} para buscar recetas por ingredientes}
    \label{sni:django-recetasIngredientes}
\end{figure}

\begin{figure}[H]
    \begin{lstlisting}[style=consola]
        def getRandom(numero : int):
            return buscador.getRandom(numero)
    \end{lstlisting}
    \caption{Método para buscar recetas aleatorias}
    \label{sni:django-aleatorias}
\end{figure}

\begin{figure}[H]
    \begin{lstlisting}[style=consola]
        def detalle_receta(request, id):
            """
            Endpoint para ver los detalles de una receta
            """
            try:
                receta = buscador.buscarRecetasID(id)
                formateado = re.sub("Paso (\d+)\n\n","",receta[2]).split(";")
                IDs = []
                if request.COOKIES.get("recetas"):
                    guardadas_json = json.loads(request.COOKIES.get("recetas"))
                    IDs  = [eval(i) for i in guardadas_json]
                
                saved = id in IDs
                context = {
                    "receta_id": receta[0],
                    "receta_nombre": receta[1],
                    "receta_pasos": formateado,
                    "receta_ingredientes": receta[4],
                    "saved": saved
                }
                return render(request,'receta.html', context)
            except:
                return redirect("index")
    \end{lstlisting}
    \caption{\gls{endpoint} para ver detalles de una receta}
    \label{sni:django-detalles}
\end{figure}

\begin{figure}[H]
    \begin{lstlisting}[style=consola]
        def guardadas(request):
            """
            Endpoint para ver las recetas guardadas
            """
            context = {}
            if request.COOKIES.get("recetas"):
                guardadas_json = json.loads(request.COOKIES.get("recetas"))
                IDs  = [eval(i) for i in guardadas_json]
                recipes = []
                for id in IDs:
                    recipe = buscador.buscarRecetasID(id)
                    saved = recipe[0] in IDs
                    recipes.append((recipe[0],recipe[1],recipe[4],saved))
        
                context = {
                    "top": buscador.getRandom(5),
                    "resultados": recipes
                }
            
            return render(request,'resultados.html',context)
    \end{lstlisting}
    \caption{\gls{endpoint} para ver las recetas guardadas del usuario}
    \label{sni:django-detalles}
\end{figure}