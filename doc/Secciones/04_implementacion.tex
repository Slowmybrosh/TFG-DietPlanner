\chapter{Implementación}
En todo proyecto existen una serie de decisiones que marcan el avance del proyecto. En este capítulo se describirán, detalladamente, las justificaciones de cada decisión que se tome durante el desarrollo del mismo. Otorgando tanto al lector como a cualquier desarrollador que busque continuar este proyecto, un entendimiento de porque se han tomado dichas decisiones.

\section{Plataforma}
Existen infinidad de plataformas para las que desarrollar software, pero a la hora de comenzar un proyecto debemos preguntarnos a cuál se orientará el software. Muchas de ellas no corresponderán con las características del proyecto, no tendría sentido desarrollar una aplicación de mapas para una televisión inteligente. 
El primer paso, es plantearse las características que tendrá nuestro software. Para este proyecto se busca que sea versátil y sencillo de utilizar, con lo cual estamos descartando muchas plataformas (relojes inteligentes, coches, televisiones, etc...). Analizaremos aquellas que más se alineen con las propiedades mencionadas: 
\begin{enumerate}
    \item Computadora
    \item Dispositivo móvil
    \item Plataforma web
\end{enumerate}
Los ordenadores cuentan con un mayor espacio de almacenamiento, mejor rendimiento. Desembocando directamente en una mayor capacidad de respuesta. Además existe un estándar en los tamaños de pantallas (y densidad de píxeles) que se utilizan, siendo capaces de adaptar mejor la interfaz de usuario. Aunque existen multitud de sistemas operativos, la mayoría de lenguajes son compatibles con multitud de sistemas. Permitiendo ejecutar la aplicación en diferentes plataformas, pero se debe tener en cuenta la posibilidad de que las llamadas a las \emph{API} de sistema pueden variar de un sistema a otro. Entre las desventajas que se han encontrado para el uso de este tipo de plataforma en el desarrollo del proyecto se hallan la falta de espacio que puede suponer utilizar el computador en la cocina, hasta los portátiles más pequeños pueden ocupar un espacio sustancial orientado a la preparación de la receta. Además, a pesar de que la posesión de un ordenador en el hogar está cada vez más extendida, según las estadísticas recopiladas por el Instituto Nacional de Estadística, solo el 82,9\% de los hogares en España cuenta con uno. El problema e esta plataforma es el nivel de conocimiento que tengan las personas de avanzada edad, que suelen sufrir de problemas de visión y no se manejan con un ordenador. Por ello, no se considera una plataforma adecuada para la aplicación. Si bien es cierto que los usuarios jóvenes si que están más familiarizados con esta tecnología, pero como se dijo anteriormente en la cocina no hay cabida para un ordenador completo.

Por otra parte el uso de un dispositivo móvil en los hogares está un poco más extendido, presente en un 99,5\% de los hogares y la mayoría de personas sabe utilizarlo fácilmente.\cite{ontsi2022} Si bien no es tan potente como un ordenador, su uso es mucho más cómodo. Pudiendo dejarlo fácilmente en cualquier rincón de la cocina o en cualquier soporte, como se haría con un libro de cocina. El principal problema es la heterogeneidad de dispositivos que existen, cada uno con una pantalla diferente, la interfaz de usuario se tendría que ajustar correctamente a cada tipo de pantalla. Y no cuenta con la misma compatibilidad que un ordenador, un claro ejemplo es el desarrollo de aplicaciones en los famosos sistemas operativos:\emph{Android} e \emph{iOS}. Los lenguajes nativos de \emph{Android} son: \emph{Java} y \emph{Kotlin}, siendo este último una mejora sustancial del primero, no seremos capaces de ejecutar una aplicación en uno de estos dos lenguajes en un dispositivo que cuente con \emph{iOS} de manera nativa. De la misma manera, no podremos ejecutar una aplicación basada en \emph{Swift}, nativa de \emph{iOS}, en un dispositivo que corra sobre \emph{Android}. No solo está orientada a una aplicación para un móvil, sino que se incluyen otros tamaños de pantalla, incluyendo también tablets. Otro problema relacionado con el usuario es el proceso de tener que descargar una aplicación, aunque el estudiante sepa descargar una aplicación con los ojos cerrados, esto puede suponer un gran reto para una persona mayor.

La última considerada es la creación de una plataforma web. Se podría considerar un híbrido entre una plataforma orientada a un ordenador y un dispositivo móvil, contando con ventajas de ambas. En primer lugar, se soluciona el problema de compatibilidad de sistemas operativos ya que se podría hacer una página web a la que acceder con una \emph{API} que acceda al \emph{backend} y todo el procesamiento esté en el lado del servidor. Mostrando al cliente los resultados en un formato web. En el caso de acceder desde un dispositivo móvil sería posible hacer una aplicación que cuente con un \emph{wrapper API} para utilizar el servicio web que se ofrece. Siendo mucho más sencillo hacer compatible la aplicación con varios sistemas operativos. La principal desventaja que se ha encontrado al uso de este tipo de plataforma son los costes derivados de utilizar un servicio ``en la nube'', teniendo no solo que costear el uso de una base de datos donde se almacenen las recetas o el uso de una \emph{API} que proporcione las recetas. Sino que habría que mantener la infraestructura de los servidores que procesan las solicitudes y los costes derivados del desarrollo de las aplicaciones multiplataforma. Esta plataforma permite que los usuarios puedan acceder a la aplicación sin tener que descargar nada y desde cualquier dispositivo. Tanto ordenadores como dispositivos móviles con cualquier sistema operativo. De cara al usuario supone una ventaja no tener que pasar por el proceso de descarga. Por ello será la plataforma elegida para esta aplicación.