\chapter{Implementación}
En todo proyecto existen una serie de decisiones que marcan el avance del proyecto. En este capítulo se describirán, detalladamente, las justificaciones de cada decisión que se tome durante el desarrollo del mismo. Otorgando tanto al lector como a cualquier desarrollador que busque continuar este proyecto, un entendimiento de porqué se han tomado dichas decisiones.

Los criterios que se seguirán para la elección de un lenguaje de programación son: 
\begin{itemize}
    \item Estándares generales: 
    \item Buenas prácticas:
    \item Deuda técnica: 
\end{itemize}

\section{Herramientas de control de versiones}
Como se explicó en capítulos anteriores, \Gls{git} es una herramienta de control de versiones muy poderosa que permite mantener una organización en los cambios realizados en el código de un proyecto por medio de \emph{issues} y \emph{commits}.

Es posible trabajar con \Gls{git} localmente, pero se consiguen mejores resultados al utilizar una plataforma en línea para almacenar estos cambios. Para mantener la coherencia en los cambios del proyecto es necesario elegir una plataforma para el control de versiones. Las dos plataformas consideradas son \href{}{Github} y \href{}{GitLab}. 

Dichas plataformas son muy parecidas, dificultando la elección de una de ellas objetivamente. Ambas permiten la gestión del proyecto por medio de un repositorio remoto y están basadas en \Gls{git}. La mayor diferencia que existe entre ambas es el objetivo para el que se utilizan. GitHub es una plataforma colaborativa que ayuda a revisar y gestionar el código remotamente. Mientras que GitLab está más enfocado a proyectos de DevOps y CI/CD. \cite{VCS2022}

En el proyecto se utiliza GitHub por ser la única herramienta de control de versiones  

\section{Herramientas de integración continua}
La integración continua (CI) es un proceso de desarrollo de software que permite al equipo de desarrollo realizar \emph{builds} consistentes, añadiendo test que permitan comprobar que las funcionalidades desarrolladas funcionen como se espera. Los tres pasos esenciales de la integración continua son: Construir, testear y mezclar. \cite{virtanen2021comparing}

Después de una investigación, de diferentes artículos\emph{online}, en el cambo de los \emph{framework} de integración continua más populares actualmente se deben valorar diferentes soluciones posibles para la implementación de la integración continua. \cite{CIKumar2023}\cite{CITaylor2023}\cite{CIRoddewig2023}

Las herramientas más repetidas son:
\begin{enumerate}
    \item Jenkins
    \item CircleCI
    \item TravisCI
    \item TeamCity
    \item GitHub Actions
\end{enumerate}

La primera herramienta que se analizará es \href{https://www.jenkins.io/}{Jenkins}, es una herramienta multiplataforma, de código abierto, ganadora de múltiples premios. Utiliza un servidor web basado en Java para ejecutar los \emph{test al software}. Soporta los siguientes lenguajes: Java, JavaScript, Groovy, GoLang, Ruby y Shell \emph{Scripting}, pero es compatible con otros lenguajes de programación como \Gls{python} haciendo uso de de una librería conocida como \href{https://python-jenkins.readthedocs.io/en/latest/index.html}{Python-Jenkins} que utiliza la \emph{REST API} de Jenkins para ejecutar los test. De acuerdo a la página oficial, ofrece las siguientes ventajas: \cite{polkhovskiy2016comparison}
\begin{itemize}
    \item Instalación y configuración sencilla
    \item Multitud de \emph{plugins} disponibles
    \item Extensibilidad
\end{itemize}

También cuenta con desventajas, por ejemplo, consume demasiados recursos cuando se empiezan a añadir muchos \emph{plugin} y trabajos.

Para concluir, Jenkins es una herramienta muy poderosa que permite a usuarios sin experiencia implementar un servicio de integración continua localmente y de manera muy sencilla. En el proyecto se busca mantener una herramienta de integración continua hospedada en la nube, que permita realizar los \emph{test} de manera dinámica.

\href{https://circleci.com/}{CircleCI}, cuenta con una instalación y configuración que no supone ninguna dificultad, \href{https://circleci.com/customers/}{muchas compañías conocidas} utilizan esta herramienta de integración continua algunas son: \href{https://www.google.com/}{Google}, \href{https://www.airbnb.es/}{AirBnb}, \href{https://www.okta.com/}{Okta}. Ofrece un plan gratuito basado en minutos, ofreciendo seis mil minutos y cinco usuarios activos por mes, contando además con diferentes planes pudiendo elegir el que más se adapte al proyecto. Ofrece las siguientes ventajas: \cite{polkhovskiy2016comparison}
\begin{itemize}
    \item Configuración rápida
    \item Personalización avanzada
    \item Es muy sencillo hacer \emph{debug} a los trabajos
    \item Cuenta con notificaciones avanzadas, soportando diferentes vías de comunicación
    \item Desarrollo continuo, permitiendo la integración con GitHub
\end{itemize}

Las opiniones de CircleCI son muy positivas, haciendo hincapié en la sencillez de la configuración utilizando un solo fichero YAML, su interfaz intuitiva que permite trabajar fácilmente con la herramienta. Por otro lado, los usuarios remarcan que si no se tiene cuidado, los créditos se acaban muy rápido además en cada \emph{push} al repositorio se inicia un nuevo \emph{test} que consume los créditos con mayor rapidez aún. \cite{CircleCI2023}

Para concluir el análisis de esta herramienta, CircleCI es una opción muy considerada para su uso en este proyecto, pero el límite de créditos mensual otorgados en el plan gratuito supone una barrera importante por la que se considera el uso de otra herramienta.

\href{https://www.travis-ci.com/}{TravisCI} es una herramienta de integración continua basada en \emph{web} construida para proyectos de código abierto, tiene la ventaja de poder utilizarse tanto localmente (\emph{on-premises}) como en la nube. Cuenta diferentes planes que para adaptarse a las necesidades del proyecto, pero lamentablemente no cuenta con un plan gratuito. 

Algunas de las ventajas que ofrece: \cite{}
\begin{itemize}
    \item Integración con GitHub
    \item Compuesto de multiples \emph{runtimes} que permiten probar aplicaciones con distintas configuraciones sin instalarlas localmente
    \item Soporta multitud de lenguajes como: Node.js, PHP, Python y Ruby
    \item Configuración sencilla por medio de un fichero YAML
\end{itemize}

Algunas opiniones de los usuarios valoran la facilidad que tienen a la hora de la configuración de la herramienta, la integración con GitHub y BitBucket y la facilidad de uso. Pero se quejan de los reportes poco claros y el precio de los planes. \cite{TravisCI2023}

Debido a que TravisCI no tiene planes gratuitos, no se escogerá esta herramienta de integración continua para el proyecto. 

\href{https://www.jetbrains.com/es-es/teamcity/}{TeamCity} es una herramienta de integración continua que permite tanto hospedar las pruebas en la nube, utilizando un agente, como hospedar la herramienta localmente reduciendo la factura de su uso. Esta herramienta ofrece una prueba gratuita \emph{on-premises} de cien \emph{builds} a partir de las cuales se comenzarán a cobrar 359 euros por \emph{build}.

Las cinco ventajas que muestra JetBrains en la página oficial de TeamCity son:
\begin{numerate}
    \item Mejora el rendimiento
    \item Configuración sencilla
    \item \emph{Tests} inteligentes
    \item Reportes en tiempo real
\end{numerate}

Otra ventaja encontrada a la hora de investigar sobre esta herramienta es el soporte de múltiples lenguajes como C++, Python, Ruby, Java, Kotlin, entre otros.

Las opiniones de los usuarios son mayormente positivas, la mayoría de usuarios agradece que sea sencillo de instalar, soporte gran cantidad de de herramientas entre ellas Docker y Kubernetes. Uno de los problemas recurrentes que se comentan es la dificultad en la curva de aprendizaje que tiene esta herramienta. 

Debido a que TeamCity es un software que no ofrece una versión gratuita y que la curva de aprendizaje es demasiado pronunciada, no se elegirá para el desarrollo del proyecto. 

Por último se analizará \href{https://github.com/features/actions}{GitHub Actions} como herramienta de integración continua. GitHub Actions permite tanto ejecutar el \emph{test} tanto en una máquina alojada en los servidores de GitHub gratuitamente como realizar el \emph{test} en localmente. La principal ventaja de esta herramienta es su integración con GitHub, permitiendo lanzar \emph{tests} cuando ocurren determinados eventos en el repositorio.

\begin{itemize}
    \item Integración con GitHub, permitiendo lanzar \emph{test} cuando ocurran determinados eventos en el repositorio, programado o un evento externo utilizando un \emph{webhook}
    \item Configuración sencilla a través de un fichero YAML
    \item Genera reportes en cada \emph{pull request}, permitiendo comprobar si el código introduce un error en el repositorio
    \item GitHub analiza el código en el repositorio y recomienda \emph{workflows} basados en el lenguaje del código
    \item Tiene una gran librería de acciones compartidas por la comunidad de GitHub
\end{itemize}

Para concluir, debido a que la ejecución gratuita de \emph{test} en los servidores de GitHub, la sencillez en la configuración del \emph{workflow} para ejecutar estos \emph{test} y la integración con GitHub siendo esta última la herramienta de control de versiones elegida, se utilizará GitHub Actions como herramienta de integración continua.





