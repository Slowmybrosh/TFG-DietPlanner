\chapter{Estado del arte}
Después de una investigación inicial en este campo buscando en diversos blogs de nutrición como \emph{HealthLine}\cite{Healthline2022} y revistas como \emph{Bussines Insider}\cite{BusinessInsider2021} y \emph{The Guardian}. Dónde se habla de aplicaciones que permiten desde desarrollar una dieta utilizando recetas publicadas por la comunidad o simplemente con los ingredientes que tenemos en nuestro frigorífico. 

Una de las más interesantes de las que se trata es \href{https://www.yummly.com}{\emph{Yummly}}, una aplicación disponible tanto en \emph{iOS} como en \emph{Android} que se adapta a las necesidades individuales de cada persona, incluyendo intolerancias alimentarias \cite{TheGuardian2016}. Esta ofrece un plan gratuito que permite buscar recetas, recomendaciones personalizadas, una lista de la compra. Además cuenta con un plan de pago que cuesta 4,99\$/mes, que incluye la creación de dietas personalizadas, información nutricional de las recetas, un motor de búsqueda por ingredientes que permite filtrar por ingredientes, alergias o tipo de dieta. Pero, desafortunadamente, no está disponible en España. 

Una aplicación quizá más conocida es \href{https://cookpad.com/es/home}{\emph{Cookpad}}, esta es una de las aplicaciones más descargadas de \emph{Apple Store} y \emph{Google Play} en la sección de cocina. Una de sus características principales es su comunidad, donde es posible compartir recetas y fotos de tus creaciones. Es posible usar esta aplicación de manera gratuita pero cuenta con algunas limitaciones. Por otra parte, la versión pro cuesta 2,99\$/mes y entre las funcionalidades que ofrece se encuentran planes de menús semanales elaborados por nutricionistas.

Otra de las aplicaciones más interesantes que se encontraron fue \href{https://www.eatthismuch.com/}{EatThisMuch}.Esta es una aplicación web, también disponible para \emp{Android} e \emph{iOS}, que permite generar una dieta de manera automática basándose en:
\begin{itemize}
    \item Las calorías que se quieren ingerir al día.
    \item El número de comidas que se desea realizar.
    \item El tipo de dieta que se quiere generar.
\end{itemize}
\emph{Eat This Much} tiene una versión gratuita abierta al usuario. Al generar la dieta es posible cambiar el plato aleatoriamente entre los que se recomiendan. Pero, el principal problema se basa en que la personalización de la dieta es muy general, sin tener en cuenta las intolerancias del usuario. De la misma manera, ofrecen una suscripción orientada a profesionales y entrenadores permitiendo que estos recomienden generen las dietas a sus clientes de manera totalmente personalizada. Esta suscripción cuesta 79,00\$/mes y solo incluye servicio para diez clientes, cada cliente extra costará 4\$/mes. 

La última aplicación de la que se hablará es \href{https://happyforks.com/}{\emph{HappyForks}}, que se trata de una herramienta que permite monitorizar una gran cantidad de valores nutricionales. Contando con una gran cantidad de utilidades para controlar una dieta ajustándose a las necesidades del usuario y una gran base de datos llena de recetas. Una de las utilidades que ofrecen permite analizar recetas y productos mostrando los ingredientes para no causar alergias al usuario. 

El software que se busca desarrollar, tiene como objetivo llegar al usuario final. Permitiendo así que cualquier persona pueda generar una dieta adaptada a sus necesidades. Por ello, se venderá un producto final bajo suscripción, cuya mensualidad vaya acorde a los gastos del desarrollo del software y el mantenimiento de la infraestructura en la nube, teniendo en cuenta un porcentaje para que haya un margen de beneficios.

En un apartado más técnico, el problema al que nos enfrentamos para la creación de una dieta es el muy conocido ``problema de la mochila", listado en los ``21 problemas NP-Completos de Richard Karp". Si el lector no conoce este problema, imagine a un autoestopista que viaja ligero de equipaje. Este tiene que elegir entre un número considerable de objetos que meter en su mochila. Debe tener en cuenta el tamaño \emph{w} del objeto \emph{j}, el espacio de la mochila \emph{c} y la utilidad \emph{p}. Al vector de objetos seleccionados se conoce como \emph{x}. Se tiene que cumplir la restricción
\begin{equation}
    \sum_{j=1}^{n}w_{j}x_{j} \leq c,
\end{equation}
y maximizar la función
\begin{equation}
    \sum_{j=1}^{n}p_{j}x_{j}.
\end{equation}

Este tipo de problemas han sido estudiados muy de cerca, tanto teórica como prácticamente. Desde el punto de vista teórico despierta interés debido a la simplicidad que permite hacer uso de una serie de propiedades combinatorias y, por otra parte, la resolución de problemas complejos subdividiéndolos en el problema tipo que se describe. En la práctica este se puede utilizar en muchas situaciones. Desde Optimizar la carga de un barco hasta planear un presupuesto capital. \cite{martello1990knapsack}

