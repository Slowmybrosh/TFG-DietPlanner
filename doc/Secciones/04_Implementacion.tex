\chapter{Implementación}
En todo proyecto existen una serie de decisiones que marcan el avance del proyecto. En este capítulo se describirán, detalladamente, las justificaciones de cada decisión que se tome durante el desarrollo del mismo. Otorgando tanto al lector como a cualquier desarrollador que busque continuar este proyecto, un entendimiento de porque se han tomado dichas decisiones.

\section{Uso de Github}
Con el objetivo de seguir una metodología ágil explicada en el capítulo anterior, se ha tomado la decisión de utilizar Github en el desarrollo de la misma. No solo se utiliza por seguir una metodología ágil, sino que al ser una plataforma para el control de versiones será posible subir el código y mantenerlo disponible para cualquier persona que quiera continuar el desarrollo al finalizar este proyecto. Conociendo desde el primer momento que pasos se han seguido para desarrollar la aplicación y porque se ha tomado cada decisión.

\section{Plataforma}
Existen infinidad de plataformas para las que desarrollar software, pero a la hora de comenzar un proyecto debemos preguntarnos a cuál se orientará el software. Muchas de ellas no corresponderán con las características del proyecto, no tendría sentido desarrollar una aplicación de mapas para una televisión inteligente. 
El primer paso, es plantearse las características que tendrá nuestro software. Para este proyecto se busca que sea versátil y sencillo de utilizar, con lo cual estamos descartando muchas plataformas (relojes inteligentes, coches, televisiones, etc...). Analizaremos aquellas que más se alineen con las propiedades mencionadas: 
\begin{enumerate}
    \item Computadora
    \item Dispositivo móvil
    \item Plataforma web
\end{enumerate}
Los ordenadores cuentan con un mayor espacio de almacenamiento, mejor rendimiento. Desembocando directamente en una mayor capacidad de respuesta. Además existe un estándar en los tamaños de pantallas (y densidad de píxeles) que se utilizan, siendo capaces de adaptar mejor la interfaz de usuario. Aunque existen multitud de sistemas operativos, la mayoría de lenguajes son compatibles con multitud de sistemas. Permitiendo ejecutar la aplicación en diferentes plataformas, pero se debe tener en cuenta la posibilidad de que las llamadas a las \emph{API} de sistema pueden variar de un sistema a otro. Entre las desventajas que se han encontrado para el uso de este tipo de plataforma en el desarrollo del proyecto se hallan la falta de espacio que puede suponer utilizar el computador en la cocina, hasta los portátiles más pequeños pueden ocupar un espacio sustancial orientado a la preparación de la receta. Además, a pesar de que la posesión de un ordenador en el hogar está cada vez más extendida, según las estadísticas recopiladas por el Instituto Nacional de Estadística, solo el 82,9\% de los hogares en España cuenta con uno.

Por otra parte el uso de un dispositivo móvil en los hogares está un poco más extendido, presente en un 99,5\% de los hogares y la mayoría de personas sabe utilizarlo fácilmente.\cite{ontsi2022} Si bien no es tan potente como un ordenador, su uso es mucho más cómodo. Pudiendo dejarlo fácilmente en cualquier rincón de la cocina o en cualquier soporte, como se haría con un libro de cocina. El principal problema es la heterogeneidad de dispositivos que existen, cada uno con una pantalla diferente, la interfaz de usuario se tendría que ajustar correctamente a cada tipo de pantalla. Y no cuenta con la misma compatibilidad que un ordenador, un claro ejemplo es el desarrollo de aplicaciones en los famosos sistemas operativos:\emph{Android} e \emph{iOS}. Los lenguajes nativos de \emph{Android} son: \emph{Java} y \emph{Kotlin}, siendo este último una mejora sustancial del primero, no seremos capaces de ejecutar una aplicación en uno de estos dos lenguajes en un dispositivo que cuente con \emph{iOS} de manera nativa. De la misma manera, no podremos ejecutar una aplicación basada en \emph{Swift}, nativa de \emph{iOS}, en un dispositivo que corra sobre \emph{Android}.

La última considerada es la creación de una plataforma web. Se podría considerar un híbrido entre una plataforma orientada a un ordenador y un dispositivo móvil, contando con ventajas de ambas. En primer lugar, se soluciona el problema de compatibilidad de sistemas operativos ya que se podría hacer una página web a la que acceder con una \emph{API} que acceda al \emph{backend} y todo el procesamiento esté en el lado del servidor. Mostrando al cliente los resultados en un formato web. En el caso de acceder desde un dispositivo móvil sería posible hacer una aplicación que cuente con un \emph{wrapper API} para utilizar el servicio web que se ofrece. Siendo mucho más sencillo hacer compatible la aplicación con varios sistemas operativos. La principal desventaja que se ha encontrado al uso de este tipo de plataforma son los costes derivados de utilizar un servicio ``en la nube'', teniendo no solo que costear el uso de una base de datos donde se almacenen las recetas o el uso de una \emph{API} que proporcione las recetas. Sino que habría que mantener la infraestructura de los servidores que procesan las solicitudes y los costes derivados del desarrollo de las aplicaciones multiplataforma.

La ventaja que decanta la decisión de elegir una plataforma web antes que una móvil se basa en que al usuario le será más sencillo utilizarla. Sin tener que descargar nada. Simplemente entrando a la página web e introduciendo los ingredientes que se quieren gastar de la despensa. Además, se podrá compatibilizar tanto para ordenador como para dispositivos móviles.

\section{Lenguaje de la plataforma}
Para desarrollar la aplicación es necesario elegir un lenguaje para el \emph{backend} que supla los requisitos para la plataforma web. Dicho lenguaje, aunque no conste como requisito de la plataforma, debe adecuarse a la habilidad del equipo de desarrollo. Con el objetivo de una mayor rápidez y sencillez a la hora de comenzar la implementación de las funcionalidades. Como pequeño análisis previo a justificar el lenguaje de programación a usar. Se estiman los siguientes requisitos: 
\begin{enumerate}
    \item Escalabilidad. Debe ser capaz de manejar el rendimiento requeridos por la plataforma. Teniendo en cuenta la velocidad de desarrollo y el rendimiento del mismo.
    \item Compatibilidad. El lenguaje debe ser compatible con diferentes frameworks y bibliotecas que permitan la construcción rápida de funcionalidades.
    \item Seguridad. Es importante que ofrezca soluciones para que la plataforma sea segura contra ataques. 
    \item Integración. El lenguaje debe ser compatible con tecnologías y servicios que se usarán en el proyecto como puede ser una base de datos o una API.
\end{enumerate}

Teniendo en cuenta estos requisitos, podemos analizar algunos lenguajes de programación ampliamente conocidos. 

Java es un lenguaje, creado con el objetivo de poder ser usado en cualquier lugar. Con independencia de la plataforma o hardware en el que se ejecutase. Debido a su gran comunidad de usuarios, tiene una gran remesa de bibliotecas disponibles para el desarrollo de aplicaciones web. Pero tiene dos problemas principales, de los que se han quejado los usuarios a lo largo de los años: Cuenta con problemas de rendimiento y problemas de seguridad. Problemas que le han costado una gran cantidad de usuarios.

JavaScript, por otro lado, se utiliza en la mayoría de páginas web para hacerlas más dinámicas. Por ello es tan conocido y usado. Permite una gran interactividad, manteniendo la sencillez a la hora de crear los elementos dinámicos tipo: botones, iconos, formularios, etc... Y se trata de un lenguaje muy rápido con gran compatibilidad.

TypeScript, es una expansión de JavaScript desarrollada por Google. Tiene el objetivo de suplir las carencias de su antecesor. Siendo un lenguaje de alto nivel, permitiendo desarrollar proyectos basados en JavaScript con mayor facilidad y agilidad. Pero está recomendado para personas que ya conocen JavaScript.

Python es el ultimo lenguaje considerado para el desarrollo de la aplicación. Se trata de uno de los lenguajes que más demanda tiene actualmente en el mercado. Su curva de aprendizaje combinadas a las habilidades del equipo, lo hacen perfecto para el desarrollo. Además es bastante escalable e integra algunos \emph{framework} orientados al desarrollo de aplicaciones web. Por ello es el lenguaje que se utilizará en el \emph{backend} de la plataforma.